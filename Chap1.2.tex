\documentclass{article}
% Change "article" to "report" to get rid of page number on title page
\usepackage{amsmath,amsfonts,amsthm,amssymb}
\usepackage{setspace}
\usepackage{Tabbing}
\usepackage{fancyhdr}
\usepackage{lastpage}
\usepackage{extramarks}
\usepackage{chngpage}
\usepackage{soul,color}
\usepackage{graphicx,float,wrapfig}
\usepackage[shortlabels]{enumitem}

% In case you need to adjust margins:
\topmargin=-0.45in      %
\evensidemargin=0in     %
\oddsidemargin=0in      %
\textwidth=6.5in        %
\textheight=9.0in       %
\headsep=0.25in         %

% Homework Specific Information
\newcommand{\Title}{Introductory Real Analysis Exercises}
\newcommand{\Assignment}{Chapter 1.2}
\newcommand{\BookAuthors}{A.Kolmogorov \& S.Fomin}
\newcommand{\AuthorName}{Max Suica}

% Setup the header and footer
\pagestyle{fancy}                                                       %
\lhead{\AuthorName}                                                     %
\chead{\Title\ : \Assignment}                                           %
\rhead{\firstxmark}                                                     %
\lfoot{\lastxmark}                                                      %
\cfoot{}                                                                %
\rfoot{Page\ \thepage\ of\ \pageref{LastPage}}                          %
\renewcommand\headrulewidth{0.4pt}                                      %
\renewcommand\footrulewidth{0.4pt}                                      %

% This is used to trace down (pin point) problems
% in latexing a document:
%\tracingall

%%%%%%%%%%%%%%%%%%%%%%%%%%%%%%%%%%%%%%%%%%%%%%%%%%%%%%%%%%%%%
% Some tools
\newcommand{\enterProblemHeader}[1]{\nobreak\extramarks{#1}{#1 continued on next page\ldots}\nobreak%
                                    \nobreak\extramarks{#1 (continued)}{#1 continued on next page\ldots}\nobreak}%
\newcommand{\exitProblemHeader}[1]{\nobreak\extramarks{#1 (continued)}{#1 continued on next page\ldots}\nobreak%
                                   \nobreak\extramarks{#1}{}\nobreak}%

\newlength{\labelLength}
\newcommand{\labelAnswer}[2]
  {\settowidth{\labelLength}{#1}%
   \addtolength{\labelLength}{0.25in}%
   \changetext{}{-\labelLength}{}{}{}%
   \noindent\begin{minipage}[c]{\columnwidth}#2\end{minipage}%
   \marginpar{#1}%

   % We put the blank space above in order to make sure this
   % \marginpar gets correctly placed.
   \changetext{}{+\labelLength}{}{}{}}%

\setcounter{secnumdepth}{0}
\newcommand{\ProblemName}{}%
\newcounter{ProblemCounter}%
\newenvironment{Problem}[1][Problem \arabic{ProblemCounter}]%
  {\stepcounter{ProblemCounter}%
   \renewcommand{\ProblemName}{#1}%
   \section{\ProblemName}%
   \enterProblemHeader{\ProblemName}}%
  {\exitProblemHeader{\ProblemName}}%

\newcommand{\problemAnswer}[1]
  {\ \\ \noindent\ \begin{minipage}[c]{\columnwidth}#1\end{minipage}}%

\newcommand{\problemLAnswer}[1]
  {\labelAnswer{\ProblemName}{#1}}

\newcommand{\SectionName}{}%
\newlength{\SectionLabelLength}{}%
\newenvironment{Section}[1]%
  {% We put this space here to make sure we're not connected to the above.
   % Otherwise the changetext can do funny things to the other margin

   \renewcommand{\SectionName}{#1}%
   \settowidth{\SectionLabelLength}{\SectionName}%
   \addtolength{\SectionLabelLength}{0.25in}%
   \changetext{}{-\SectionLabelLength}{}{}{}%
   \subsection{\SectionName}%
   \enterProblemHeader{\ProblemName\ [\SectionName]}}%
  {\enterProblemHeader{\ProblemName}%

   % We put the blank space above in order to make sure this margin
   % change doesn't happen too soon (otherwise \sectionAnswer's can
   % get ugly about their \marginpar placement.
   \changetext{}{+\SectionLabelLength}{}{}{}}%

\newcommand{\sectionAnswer}[1]
  {% We put this space here to make sure we're disconnected from the previous
   % passage

   \noindent\begin{minipage}[c]{\columnwidth}#1\end{minipage}%
   \enterProblemHeader{\ProblemName}\exitProblemHeader{\ProblemName}%
   \marginpar{\SectionName}%

   % We put the blank space above in order to make sure this
   % \marginpar gets correctly placed.
   }%

%%%%%%%%%%%%%%%%%%%%%%%%%%%%%%%%%%%%%%%%%%%%%%%%%%%%%%%%%%%%%


%%%%%%%%%%%%%%%%%%%%%%%%%%%%%%%%%%%%%%%%%%%%%%%%%%%%%%%%%%%%%
% Make title
\title{\vspace{2in}\textmd{\textbf{\Title}}
\\\normalsize\vspace{0.1in}{\BookAuthors}
\\\vspace{0.1in}\large{\textit{\ }}\vspace{3in}}
\date{}
\author{\textbf{\AuthorName}}
%%%%%%%%%%%%%%%%%%%%%%%%%%%%%%%%%%%%%%%%%%%%%%%%%%%%%%%%%%%%%

\begin{document}
\begin{spacing}{1.1}

\maketitle
\newpage
% Uncomment the \tableofcontents and \newpage lines to get a Contents page
% Uncomment the \setcounter line as well if you do NOT want subsections
%       listed in Contents
%\setcounter{tocdepth}{1}
%\tableofcontents
%\newpage

% When problems are long, it may be desirable to put a \newpage or a
% \clearpage before each Problem environment

\clearpage

%%%%%%%%%%%%%%%%%%%%%%%%%%%%%%%%%%%%%%%%%%%%%%%%%%%%%%%%%%%%
\section{\Assignment\ Exercises}


\begin{Problem}
Prove that a set with an uncountable subset is itself uncountable.

\problemAnswer{
Given a set $M$ with an uncountable subset $A$ we can see that any counting of $M$ must eventually count all the elements of $A$, which is impossible.
}
\end{Problem}


\begin{Problem}
Let $M$ be any infinite set and $A$ any countable set. Prove that $M \sim M \cup A.$

\problemAnswer{
Let $M' = \{m_1, m2, m3 \ldots\}$ be a countable subset of $M$ and write $A =\{a_1, a_2, a_3, \ldots\}$. Define $f:M \rightarrow M \cup A$ as a mapping that carries $M - M'$ into itself and maps $M'$ thusly
$$f(m_k) = \left\{ 
\begin{array}{l l}
  m_{(k+1)/2} & \quad \text{if $k$ is odd}\\
  a_{(k/2)} & \quad \text{if $k$ is even}\\
\end{array} \right.$$
Then $f$ is a one to one mapping.
}
\end{Problem}


\begin{Problem}
Prove that each of the following sets is countable:

\begin{enumerate}
\item The set of all numbers with two distinct decimal expansions (like 0.50000  and 0.4999). 
\item The set of all rational points in the plane.
\item The set of all rational intervals.
\item The set of all polynomials with rational coefficients.
\end{enumerate}

\problemAnswer{
\begin{enumerate}
	\item The set of numbers with two decimal expansions is exactly the set with finite decimal expansions, which we can separated according to the length of their finite expansions. Since the the set numbers of a certain length is countable for each finite length, the numbers are all countable.
	\item This is equivalent to $\mathbb{Q}^2$.
	\item This is equivalent to a subset of $\mathbb{Q}^2$, namely $\{(a,b)\in\mathbb{Q}^2 \mid a \leq b\}.$
	\item This is equivalent to $\mathbb{Q}^\omega$, i.e., a countable product of countable sets.
\end{enumerate}
}
\end{Problem}

\begin{Problem}
A number $\alpha$ is called \textit{algebraic} if it is the roof of a polynomial equation with rational coefficients. Prove that the set of all algebraic numbers is countable.

\problemAnswer{
Since the set of rational polynomials is countable and they each have a finite number of roots, the union of the roots of these polynomials must also be countable.
}
\end{Problem}

\begin{Problem}
Prove the existence of uncountably many \textit{transcendental} numbers.

\problemAnswer{
The real numbers are uncountable due to an argument of Cantor, but there are only countably many algebraic numbers. Therefore the transcendental numbers must be uncountable in cardinality.
}
\end{Problem}

\begin{Problem}
Prove that the set of all real functions defined on a set $M$ is of power greater than the power of $M$. In particular, prove that the power of the set of all real functions defined in the unit interval is greater than \textit{c}.

\problemAnswer{
The set $2^M$ of characteristic functions on $M$ is of power greater than $M$. But since the characteristic functions correspond to a subset of the real functions on $M$, the power of the real functions is greater than that of $M$. Specifically, the set of real functions on $[0,1]$ must be of greater power than \textit{c}.
}
\end{Problem}

\begin{Problem}
Give an indirect proof of the equivalence of the closed interval $[a,b]$, the open interval $(a,b)$, and the half open interval $(a,b]$ or $[b,a)$.

\problemAnswer{
We can construct an injection from any of these intervals into the others using linear scaling. But then due to Cantor-Bernstein we have that they are all equivalent.
}
\end{Problem}

\begin{Problem}
Prove that the union of a finite or countable number of set each of power \textit{c} is itself of power \textit{c}.

\problemAnswer{
We may divide the continuum into a countable number of subsets with power equal to the whole. Since each of these subsets corresponds to one of the sets in the union we have a surjection from the continuum to the union so its power is no greater than that of the continuum. Since it is a union, its power must be at least that of one of its elements, which is that of the continuum.
}
\end{Problem}

\begin{Problem}
Prove that each of the following sets has the power of the continuum:

\begin{enumerate}[a)]
	\item The set of all infinite sequences of positive integers.
	\item The set of all ordered \textit{n}-tuples of real numbers.
	\item The set of all infinite sequences of real numbers.
\end{enumerate}

\problemAnswer{
\begin{enumerate}[a)]
	\item By a diagonalization agrument, this set is uncountable. To see that it is power no greater than the continuum we observe that each infinite sequence $x = \{a_k\}_{k=1}^\infty$ can be mapped into the $[0,1]$ by writing each $a_k$ in base $b$, and concatenating the digits of these numbers in a base $b+1$ decimal sequence, using the digit $b+1$ as a delimiter: 
	$$x' = 0.a_{11}a_{12}\ldots a_{1n_1}(b+1)a_{21}a_{22}\ldots a_{2n_2}(b+1)\ldots a_{k1}a_{k2}\ldots a_{kn_k}(b+1)\ldots$$
	Thus it is of power \textit{c}.
  \item This is a finite product of $\mathbb{R}$, which has the same power as $mathbb{r}$.
  \item This is a countable product of $\mathbb{R}$, which must have the same power.
\end{enumerate}

}
\end{Problem}

\begin{Problem}
Develop a contradiction inherent in the notion of the "set of all sets which are not members of themselves."

\problemAnswer{
Given such set $\mathcal{M}$, one might ask whether it possesses the property "$\mathcal{M}$ is not member of itself". Since it is the set of sets that have this property, it must contain itself if and only if doesn't contain itself, which is contradictory.
}
\end{Problem}

\end{spacing}
\end{document}

%%%%%%%%%%%%%%%%%%%%%%%%%%%%%%%%%%%%%%%%%%%%%%%%%%%%%%%%%%%%%
