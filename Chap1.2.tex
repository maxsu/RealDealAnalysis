\documentclass{article}
% Change "article" to "report" to get rid of page number on title page
\usepackage{amsmath,amsfonts,amsthm,amssymb}
\usepackage{setspace}
\usepackage{Tabbing}
\usepackage{fancyhdr}
\usepackage{lastpage}
\usepackage{extramarks}
\usepackage{chngpage}
\usepackage{soul,color}
\usepackage{graphicx,float,wrapfig}
\usepackage[shortlabels]{enumitem}

% In case you need to adjust margins:
\topmargin=-0.45in      %
\evensidemargin=0in     %
\oddsidemargin=0in      %
\textwidth=6.5in        %
\textheight=9.0in       %
\headsep=0.25in         %

% Homework Specific Information
\newcommand{\Title}{Introductory Real Analyis Exercises}
\newcommand{\Assignment}{Chaper 1.2}
\newcommand{\BookAuthors}{A.Kolmogorov \& S.Fomin}
\newcommand{\AuthorName}{Max Suica}

% Setup the header and footer
\pagestyle{fancy}                                                       %
\lhead{\AuthorName}                                                     %
\chead{\Title\ : \Assignment}                                           %
\rhead{\firstxmark}                                                     %
\lfoot{\lastxmark}                                                      %
\cfoot{}                                                                %
\rfoot{Page\ \thepage\ of\ \pageref{LastPage}}                          %
\renewcommand\headrulewidth{0.4pt}                                      %
\renewcommand\footrulewidth{0.4pt}                                      %

% This is used to trace down (pin point) problems
% in latexing a document:
%\tracingall

%%%%%%%%%%%%%%%%%%%%%%%%%%%%%%%%%%%%%%%%%%%%%%%%%%%%%%%%%%%%%
% Some tools
\newcommand{\enterProblemHeader}[1]{\nobreak\extramarks{#1}{#1 continued on next page\ldots}\nobreak%
                                    \nobreak\extramarks{#1 (continued)}{#1 continued on next page\ldots}\nobreak}%
\newcommand{\exitProblemHeader}[1]{\nobreak\extramarks{#1 (continued)}{#1 continued on next page\ldots}\nobreak%
                                   \nobreak\extramarks{#1}{}\nobreak}%

\newlength{\labelLength}
\newcommand{\labelAnswer}[2]
  {\settowidth{\labelLength}{#1}%
   \addtolength{\labelLength}{0.25in}%
   \changetext{}{-\labelLength}{}{}{}%
   \noindent\fbox{\begin{minipage}[c]{\columnwidth}#2\end{minipage}}%
   \marginpar{\fbox{#1}}%

   % We put the blank space above in order to make sure this
   % \marginpar gets correctly placed.
   \changetext{}{+\labelLength}{}{}{}}%

\setcounter{secnumdepth}{0}
\newcommand{\ProblemName}{}%
\newcounter{ProblemCounter}%
\newenvironment{Problem}[1][Problem \arabic{ProblemCounter}]%
  {\stepcounter{ProblemCounter}%
   \renewcommand{\ProblemName}{#1}%
   \section{\ProblemName}%
   \enterProblemHeader{\ProblemName}}%
  {\exitProblemHeader{\ProblemName}}%

\newcommand{\problemAnswer}[1]
  {\ \\ \noindent\fbox{\begin{minipage}[c]{\columnwidth}#1\end{minipage}}}%

\newcommand{\problemLAnswer}[1]
  {\labelAnswer{\ProblemName}{#1}}

\newcommand{\SectionName}{}%
\newlength{\SectionLabelLength}{}%
\newenvironment{Section}[1]%
  {% We put this space here to make sure we're not connected to the above.
   % Otherwise the changetext can do funny things to the other margin

   \renewcommand{\SectionName}{#1}%
   \settowidth{\SectionLabelLength}{\SectionName}%
   \addtolength{\SectionLabelLength}{0.25in}%
   \changetext{}{-\SectionLabelLength}{}{}{}%
   \subsection{\SectionName}%
   \enterProblemHeader{\ProblemName\ [\SectionName]}}%
  {\enterProblemHeader{\ProblemName}%

   % We put the blank space above in order to make sure this margin
   % change doesn't happen too soon (otherwise \sectionAnswer's can
   % get ugly about their \marginpar placement.
   \changetext{}{+\SectionLabelLength}{}{}{}}%

\newcommand{\sectionAnswer}[1]
  {% We put this space here to make sure we're disconnected from the previous
   % passage

   \noindent\fbox{\begin{minipage}[c]{\columnwidth}#1\end{minipage}}%
   \enterProblemHeader{\ProblemName}\exitProblemHeader{\ProblemName}%
   \marginpar{\fbox{\SectionName}}%

   % We put the blank space above in order to make sure this
   % \marginpar gets correctly placed.
   }%

%%%%%%%%%%%%%%%%%%%%%%%%%%%%%%%%%%%%%%%%%%%%%%%%%%%%%%%%%%%%%


%%%%%%%%%%%%%%%%%%%%%%%%%%%%%%%%%%%%%%%%%%%%%%%%%%%%%%%%%%%%%
% Make title
\title{\vspace{2in}\textmd{\textbf{\Title}}
\\\normalsize\vspace{0.1in}{\BookAuthors}
\\\vspace{0.1in}\large{\textit{\ }}\vspace{3in}}
\date{}
\author{\textbf{\AuthorName}}
%%%%%%%%%%%%%%%%%%%%%%%%%%%%%%%%%%%%%%%%%%%%%%%%%%%%%%%%%%%%%

\begin{document}
\begin{spacing}{1.1}

\maketitle
\newpage
% Uncomment the \tableofcontents and \newpage lines to get a Contents page
% Uncomment the \setcounter line as well if you do NOT want subsections
%       listed in Contents
%\setcounter{tocdepth}{1}
%\tableofcontents
%\newpage

% When problems are long, it may be desirable to put a \newpage or a
% \clearpage before each Problem environment

\clearpage

%%%%%%%%%%%%%%%%%%%%%%%%%%%%%%%%%%%%%%%%%%%%%%%%%%%%%%%%%%%%
\section{\Assignment\ Exercises}


\begin{Problem}
Prove that a set with an uncountable subset is itself uncountable.

\problemAnswer{
Given a set $M$ with an uncountable subset $A$ we can see that the set $M$ is uncountable since any counting of $M$ must eventually count all the elements of $A$, which is impossible.
}
\end{Problem}


\begin{Problem}
Let $M$ be any infinite set and $A$ any countable set. Prove that $M ~ M \cup A.$

\problemAnswer{

}
\end{Problem}


\begin{Problem}
Prove that each of the following sets is countable:

%begin{enumerate}
%\item The set of all numbers with two distinct decimal expansions (like 0.50000  and 0.4999 . 
%\item The set of akk rational points in the plane.
%\item The set of all rational intervals.
%\item The set of all polynomials with rational coefficients.
%\end{enumerate}

\problemAnswer{

}
\end{Problem}

\begin{Problem}
A number $\alpha$ is called \textit{algebraic} if it is the roof of a polynomial equation with rational coefficients. Prove that the set of all algebraic numbers is countable.

\problemAnswer{

}
\end{Problem}

\begin{Problem}
Prove the existance of uncountably many \textit{transcendental} numbers.

\problemAnswer{

}
\end{Problem}

\begin{Problem}
Prove that the set of all real functions defined on a set $M$ is of power greater than the power of $M$. In particular, prove that the power of the set of all real functions defined in the unit interval is greater tha \textit{c}. (Use the fact that the set of charachteristic functions on $M$ is equivalent to the set of all subsets of $M$.

\problemAnswer{

}
\end{Problem}

\begin{Problem}
Give an indirect proof of the equivalence of the closed interval $[a,b]$, the open interval $(a,b)$, and the half open interval $(a,b]$ or $[b,a)$.

\problemAnswer{

}
\end{Problem}

\begin{Problem}
Prove that the union of a finite or countable number of set each of power \textit{c} is itself of power \textit{c}.

\problemAnswer{

}
\end{Problem}

\begin{Problem}
Proce that each of the following sets has the power of the continuum:

\begin{enumerate}[a)]
	\item The set of all infinite sequences of positive integers.
	\item The set of all ordered \textit{n}-tuples of real numbers.
	\item The set of all infinite sequences of real numbers.
\end{enumerate}

\problemAnswer{

}
\end{Problem}

\begin{Problem}
Develop a contradiction inherent in the notion of the "set of all sets which are not members of themselves."'

\problemAnswer{

}
\end{Problem}

\end{spacing}
\end{document}

%%%%%%%%%%%%%%%%%%%%%%%%%%%%%%%%%%%%%%%%%%%%%%%%%%%%%%%%%%%%%
