\documentclass{article}
% Change "article" to "report" to get rid of page number on title page
\usepackage{amsmath,amsfonts,amsthm,amssymb}
\usepackage{setspace}
\usepackage{Tabbing}
\usepackage{fancyhdr}
\usepackage{lastpage}
\usepackage{extramarks}
\usepackage{chngpage}
\usepackage{soul,color}
\usepackage{graphicx,float,wrapfig}
\usepackage[shortlabels]{enumitem}
\usepackage{mathrsfs}

% In case you need to adjust margins:
\topmargin=-0.45in      %
\evensidemargin=0in     %
\oddsidemargin=0in      %
\textwidth=6.5in        %
\textheight=9.0in       %
\headsep=0.25in         %

% Homework Specific Information
\newcommand{\Title}{Introductory Real Analyis Exercises}
\newcommand{\Assignment}{Chaper 1.4}
\newcommand{\BookAuthors}{A.Kolmogorov \& S.Fomin}
\newcommand{\AuthorName}{Max Suica}

% Setup the header and footer
\pagestyle{fancy}                                                       %
\lhead{\AuthorName}                                                     %
\chead{\Title\ : \Assignment}                                           %
\rhead{\firstxmark}                                                     %
\lfoot{\lastxmark}                                                      %
\cfoot{}                                                                %
\rfoot{Page\ \thepage\ of\ \pageref{LastPage}}                          %
\renewcommand\headrulewidth{0.4pt}                                      %
\renewcommand\footrulewidth{0.4pt}                                      %

% This is used to trace down (pin point) problems
% in latexing a document:
%\tracingall

%%%%%%%%%%%%%%%%%%%%%%%%%%%%%%%%%%%%%%%%%%%%%%%%%%%%%%%%%%%%%
% Some tools
\newcommand{\enterProblemHeader}[1]{\nobreak\extramarks{#1}{#1 continued on next page\ldots}\nobreak%
                                    \nobreak\extramarks{#1 (continued)}{#1 continued on next page\ldots}\nobreak}%
\newcommand{\exitProblemHeader}[1]{\nobreak\extramarks{#1 (continued)}{#1 continued on next page\ldots}\nobreak%
                                   \nobreak\extramarks{#1}{}\nobreak}%

\newlength{\labelLength}
\newcommand{\labelAnswer}[2]
  {\settowidth{\labelLength}{#1}%
   \addtolength{\labelLength}{0.25in}%
   \changetext{}{-\labelLength}{}{}{}%
   \noindent\begin{minipage}[c]{\columnwidth}#2\end{minipage}%
   \marginpar{#1}%

   % We put the blank space above in order to make sure this
   % \marginpar gets correctly placed.
   \changetext{}{+\labelLength}{}{}{}}%

\setcounter{secnumdepth}{0}
\newcommand{\ProblemName}{}%
\newcounter{ProblemCounter}%
\newenvironment{Problem}[1][Problem \arabic{ProblemCounter}]%
  {\stepcounter{ProblemCounter}%
   \renewcommand{\ProblemName}{#1}%
   \section{\ProblemName}%
   \enterProblemHeader{\ProblemName}}%
  {\exitProblemHeader{\ProblemName}}%

\newcommand{\problemAnswer}[1]
  {\ \\ \noindent\ \begin{minipage}[c]{\columnwidth}#1\end{minipage}}%

\newcommand{\problemLAnswer}[1]
  {\labelAnswer{\ProblemName}{#1}}

\newcommand{\SectionName}{}%
\newlength{\SectionLabelLength}{}%
\newenvironment{Section}[1]%
  {% We put this space here to make sure we're not connected to the above.
   % Otherwise the changetext can do funny things to the other margin

   \renewcommand{\SectionName}{#1}%
   \settowidth{\SectionLabelLength}{\SectionName}%
   \addtolength{\SectionLabelLength}{0.25in}%
   \changetext{}{-\SectionLabelLength}{}{}{}%
   \subsection{\SectionName}%
   \enterProblemHeader{\ProblemName\ [\SectionName]}}%
  {\enterProblemHeader{\ProblemName}%

   % We put the blank space above in order to make sure this margin
   % change doesn't happen too soon (otherwise \sectionAnswer's can
   % get ugly about their \marginpar placement.
   \changetext{}{+\SectionLabelLength}{}{}{}}%

\newcommand{\sectionAnswer}[1]
  {% We put this space here to make sure we're disconnected from the previous
   % passage

   \noindent\begin{minipage}[c]{\columnwidth}#1\end{minipage}%
   \enterProblemHeader{\ProblemName}\exitProblemHeader{\ProblemName}%
   \marginpar{\SectionName}%

   % We put the blank space above in order to make sure this
   % \marginpar gets correctly placed.
   }%

%%%%%%%%%%%%%%%%%%%%%%%%%%%%%%%%%%%%%%%%%%%%%%%%%%%%%%%%%%%%%


%%%%%%%%%%%%%%%%%%%%%%%%%%%%%%%%%%%%%%%%%%%%%%%%%%%%%%%%%%%%%
% Make title
\title{\vspace{2in}\textmd{\textbf{\Title}}
\\\normalsize\vspace{0.1in}{\BookAuthors}
\\\vspace{0.1in}\large{\textit{\ }}\vspace{3in}}
\date{}
\author{\textbf{\AuthorName}}
%%%%%%%%%%%%%%%%%%%%%%%%%%%%%%%%%%%%%%%%%%%%%%%%%%%%%%%%%%%%%

\begin{document}
\begin{spacing}{1.1}

\maketitle
\newpage
% Uncomment the \tableofcontents and \newpage lines to get a Contents page
% Uncomment the \setcounter line as well if you do NOT want subsections
%       listed in Contents
%\setcounter{tocdepth}{1}
%\tableofcontents
%\newpage

% When problems are long, it may be desirable to put a \newpage or a
% \clearpage before each Problem environment

\clearpage

%%%%%%%%%%%%%%%%%%%%%%%%%%%%%%%%%%%%%%%%%%%%%%%%%%%%%%%%%%%%

\section{\Assignment\ Exercises}


\begin{Problem}
Let $X$ be an uncountable set, and let $\mathscr{R}$ be the ring consisting of all finite subsets of $X$ and their complements. Is $\mathscr{R}$ a $\sigma$-ring?

\problemAnswer{
No. If $\mathscr{R}$ were a $\sigma$-ring then we could construct the union $\displaystyle{A_\infty = \bigcup^\infty A_k}$ of an infinite sequence  of mutually disjoint finite subsets of $X$. $A_\infty$ would be a countably infinite subset of $X$, and so neither a finite subset, nor a complement of such a finite subset, which would necessarily be uncountable. The ring $\mathcal{R}$ by its definition does not have these kinds of elements, thus it is not a $\sigma$-ring.
}
\end{Problem}
\begin{Problem}
Are open intervals Borel sets?

\problemAnswer{
Any open interval $(a,b)$ may be constructed like this
\[(a,b) = \Bigl([a,a] \bigtriangleup [a,b]\Bigr)\cap\Bigl([a,b] \bigtriangleup [b,b]\Bigr)\]
 
and so are included in the in the Borel-algebra generated by the the closed intervals.
}
\end{Problem}

\begin{Problem}
Let $y = f(x)$ be a function defined on a set $M$ and taking values in a set $N$. Let $\mathcal{M}$ be a system of subsets of $M$, and let $f(\mathcal{M})$ denote the system of all images $f(A)$ of sets $A \in \mathcal{M}$. Moreover, let $\mathcal{N}$ be a system of subsets of $N$, and let $f^{-1}(\mathcal{N})$ denote the system of all preimages $f^{-1}(B)$ of sets $B \in \mathcal{N}$. Prove that
\begin{enumerate}[a)]
	\item If $\mathcal{N}$ is a ring, so is $f^{-1}(\mathcal{N})$.
	\item If $\mathcal{N}$ is an algebra, so is $f^{-1}(\mathcal{N})$.
	\item If $\mathcal{N}$ is a B-algebra, so is $f^{-1}(\mathcal{N})$.
	\item $\mathscr{R}(f^{-1}(\mathcal{N}))=f^{-1}(\mathscr{R}(\mathcal{N}))$.
	\item $\mathscr{B}(f^{-1}(\mathcal{N}))=f^{-1}(\mathscr{B}(\mathcal{N}))$.
\end{enumerate}
\ \\
Which of these assertions remain true if $\mathcal{N}$ is replaced by $\mathcal{M}$ and $f^{-1}$ by $f$?

\problemAnswer{
First a few rules to remember. Let $g = f^{-1}$ be a (possibly one to many) function . Then
\[g\left(\bigcup_\alpha A_\alpha\right)= \bigcup_\alpha g(A_\alpha),\]
\[g(A\cap B) = g(A)\cap g(B),\]
\[g(A \bigtriangleup B) = g(A)\bigtriangleup g(B).\]

\begin{enumerate}[a)]
	\item It's easy to see that for every $A, B \in \mathcal{N}$, $f^{-1}(A) \cap f^{-1}(B)$ and $f^{-1}(A) \bigtriangleup f^{-1}(B)$ exist in $f^{-1}(\mathscr{N})$. They are, respectively, $f^{-1}(A\cap B)$ and $f^{-1}(A \bigtriangleup B)$. Thus $f^{-1}(\mathcal{N})$ is a ring.
	\item If $\mathcal{N}$ contains a unit $E$, then 
	\[f^{-1}(E) = \bigcup_{A\in\mathcal{N}}f^{-1}(A)\]
	 must be the unit of $f^{-1}(\mathcal{N})$, which is thusly an algebra.
	\item Any countable union of sets $\bigcup_\alpha A_\alpha \in \mathcal{N}$ will correspond to a countable union of sets $\bigcup_\alpha f^{-1}(A_\alpha) \in f^{-1}(\mathcal{N})$, so it is a B-algebra.
	\item 
	\item
\end{enumerate}
}
\end{Problem}

\end{spacing}
\end{document}

%%%%%%%%%%%%%%%%%%%%%%%%%%%%%%%%%%%%%%%%%%%%%%%%%%%%%%%%%%%%%
