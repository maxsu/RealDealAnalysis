\documentclass{article}
% Change "article" to "report" to get rid of page number on title page
\usepackage{amsmath,amsfonts,amsthm,amssymb}
\usepackage{setspace}
\usepackage{Tabbing}
\usepackage{fancyhdr}
\usepackage{lastpage}
\usepackage{extramarks}
\usepackage{chngpage}
\usepackage{soul,color}
\usepackage{graphicx,float,wrapfig}
\usepackage[shortlabels]{enumitem}

% In case you need to adjust margins:
\topmargin=-0.45in      %
\evensidemargin=0in     %
\oddsidemargin=0in      %
\textwidth=6.5in        %
\textheight=9.0in       %
\headsep=0.25in         %

% Homework Specific Information
\newcommand{\Title}{Introductory Real Analyis Exercises}
\newcommand{\Assignment}{Chaper 1.1}
\newcommand{\BookAuthors}{A.Kolmogorov \& S.Fomin}
\newcommand{\AuthorName}{Max Suica}

% Setup the header and footer
\pagestyle{fancy}                                                       %
\lhead{\AuthorName}                                                     %
\chead{\Title\ : \Assignment}                                           %
\rhead{\firstxmark}                                                     %
\lfoot{\lastxmark}                                                      %
\cfoot{}                                                                %
\rfoot{Page\ \thepage\ of\ \pageref{LastPage}}                          %
\renewcommand\headrulewidth{0.4pt}                                      %
\renewcommand\footrulewidth{0.4pt}                                      %

% This is used to trace down (pin point) problems
% in latexing a document:
%\tracingall

%%%%%%%%%%%%%%%%%%%%%%%%%%%%%%%%%%%%%%%%%%%%%%%%%%%%%%%%%%%%%
% Some tools
\newcommand{\enterProblemHeader}[1]{\nobreak\extramarks{#1}{#1 continued on next page\ldots}\nobreak%
                                    \nobreak\extramarks{#1 (continued)}{#1 continued on next page\ldots}\nobreak}%
\newcommand{\exitProblemHeader}[1]{\nobreak\extramarks{#1 (continued)}{#1 continued on next page\ldots}\nobreak%
                                   \nobreak\extramarks{#1}{}\nobreak}%

\newlength{\labelLength}
\newcommand{\labelAnswer}[2]
  {\settowidth{\labelLength}{#1}%
   \addtolength{\labelLength}{0.25in}%
   \changetext{}{-\labelLength}{}{}{}%
   \noindent\begin{minipage}[c]{\columnwidth}#2\end{minipage}%
   \marginpar{#1}%

   % We put the blank space above in order to make sure this
   % \marginpar gets correctly placed.
   \changetext{}{+\labelLength}{}{}{}}%

\setcounter{secnumdepth}{0}
\newcommand{\ProblemName}{}%
\newcounter{ProblemCounter}%
\newenvironment{Problem}[1][Problem \arabic{ProblemCounter}]%
  {\stepcounter{ProblemCounter}%
   \renewcommand{\ProblemName}{#1}%
   \section{\ProblemName}%
   \enterProblemHeader{\ProblemName}}%
  {\exitProblemHeader{\ProblemName}}%

\newcommand{\problemAnswer}[1]
  {\ \\ \noindent\ \begin{minipage}[c]{\columnwidth}#1\end{minipage}}%

\newcommand{\problemLAnswer}[1]
  {\labelAnswer{\ProblemName}{#1}}

\newcommand{\SectionName}{}%
\newlength{\SectionLabelLength}{}%
\newenvironment{Section}[1]%
  {% We put this space here to make sure we're not connected to the above.
   % Otherwise the changetext can do funny things to the other margin

   \renewcommand{\SectionName}{#1}%
   \settowidth{\SectionLabelLength}{\SectionName}%
   \addtolength{\SectionLabelLength}{0.25in}%
   \changetext{}{-\SectionLabelLength}{}{}{}%
   \subsection{\SectionName}%
   \enterProblemHeader{\ProblemName\ [\SectionName]}}%
  {\enterProblemHeader{\ProblemName}%

   % We put the blank space above in order to make sure this margin
   % change doesn't happen too soon (otherwise \sectionAnswer's can
   % get ugly about their \marginpar placement.
   \changetext{}{+\SectionLabelLength}{}{}{}}%

\newcommand{\sectionAnswer}[1]
  {% We put this space here to make sure we're disconnected from the previous
   % passage

   \noindent\begin{minipage}[c]{\columnwidth}#1\end{minipage}%
   \enterProblemHeader{\ProblemName}\exitProblemHeader{\ProblemName}%
   \marginpar{\SectionName}%

   % We put the blank space above in order to make sure this
   % \marginpar gets correctly placed.
   }%

%%%%%%%%%%%%%%%%%%%%%%%%%%%%%%%%%%%%%%%%%%%%%%%%%%%%%%%%%%%%%


%%%%%%%%%%%%%%%%%%%%%%%%%%%%%%%%%%%%%%%%%%%%%%%%%%%%%%%%%%%%%
% Make title
\title{\vspace{2in}\textmd{\textbf{\Title}}
\\\normalsize\vspace{0.1in}{\BookAuthors}
\\\vspace{0.1in}\large{\textit{\ }}\vspace{3in}}
\date{}
\author{\textbf{\AuthorName}}
%%%%%%%%%%%%%%%%%%%%%%%%%%%%%%%%%%%%%%%%%%%%%%%%%%%%%%%%%%%%%

\begin{document}
\begin{spacing}{1.1}

\maketitle
\newpage
% Uncomment the \tableofcontents and \newpage lines to get a Contents page
% Uncomment the \setcounter line as well if you do NOT want subsections
%       listed in Contents
%\setcounter{tocdepth}{1}
%\tableofcontents
%\newpage

% When problems are long, it may be desirable to put a \newpage or a
% \clearpage before each Problem environment

\clearpage

%%%%%%%%%%%%%%%%%%%%%%%%%%%%%%%%%%%%%%%%%%%%%%%%%%%%%%%%%%%%

\section{\Assignment\ Exercises}


\begin{Problem} Prove that if $A \cup B = A$ and $A \cap B = A$ then $A = B$.\\

\problemAnswer{
To show this we observe that $A \cup B = A$ implies $B \subset A$, while $A \cap B = A$ implies $A \subset B$. But then we have the definition of equality of two sets and $A = B$.
}
\end{Problem}


\begin{Problem} Show that in general $(A - B) \cup B \ne B$.\\

\problemAnswer{
Suppose that $A$ is not a subset of $B$. Then $(A - B)$ contains at least one element contained in $A$ alone. Thus the expression $(A - B) \cup B \ne B$ is valid in the general case.
}
\end{Problem}


\begin{Problem}
Let $ A = \{2,4,...,2n,...\}$ and $B =\{3,6,...,3n,...\}$. Find $A \cap B$ and $A - B$.\\

\problemAnswer{
$$A \cap B = \{a \in \mathbb{N} \mid (2 \mid a \wedge 3 \mid a)\} = \{6,12,...,6n,...\}$$
$$A-B = \{a \in \mathbb{N} \mid (2 \mid a \wedge 3\nmid a)\} = \{2,4,8,10,..., 6n-4, 6n-2 \}$$
}
\end{Problem}


\begin{Problem}
Prove that
\begin {enumerate}[a)]
 \item a) $(A-B) \cap C = (A \cap C) - (B \cap C))$
 \item$ A \bigtriangleup B = (A \cup B) - (A \cap B)$
\end{enumerate}

\problemAnswer{
\begin{enumerate}[a)]

  \item We use the notation $AB$ to denote intersections and $A+B$ to denote unions, and $A^c$ to denote the complement of $A$ relative to some set $\Omega$ which is a superset of the sets in our discourse. Then the difference between two sets may be written as an intersection, $A - B = AB^c$. Then
$$(A-B) \cap C = AB^cC = AB^cC + \emptyset = ACB^c + ACC^c = AC(B^c + C^c) = AC(BC)^c = (A \cap C) - (B \cap C) $$

  \item To show this we state the definition $A \bigtriangleup B = AB^c + A^cB$. Then
$$ (A \cup B) - (A \cap B) = (A+B)(AB)^c = (A+B)(A^c+B^c) = AA^c + AB^c + A^cB + BB^c = AB^c + A^cB = A \bigtriangleup B $$
\end{enumerate}
}
%% I would like to annotate this with rules, either underneath each step, or as a
%% Progressive column of equations and reasons.
\end{Problem}


\begin{Problem}
Prove that
$$\bigcup_\alpha A_\alpha - \bigcup_\alpha B_\alpha \subset \bigcup_\alpha (A_\alpha - B_\alpha)$$\\

\problemAnswer{
First we rewrite the problem as
$$\left(\bigcup_\alpha A_\alpha\right) \cap \left(\bigcup_\alpha B_\alpha\right)^c \subset \bigcup_\alpha \left( A_\alpha \cap B_\alpha^c \right) $$
Through deMorgan's law and the distributive property we see that
$$\left(\bigcup_\alpha A_\alpha\right) \cap \left(\bigcup_\alpha B_\alpha\right)^c =
  \bigcup_\alpha \left(A_\alpha\ \cap \ \bigcap_\alpha B_\alpha^c\right).$$
Here it's clear that $\bigcap_\alpha B_\alpha^c \subset B_\alpha^c$ for each, $\alpha$. But then $A_\alpha \cap (\bigcap_\alpha B_\alpha^c)$ is a subset of  $A_\alpha \cap B_\alpha^c$ for each $\alpha$, proving the formula.
}
\end{Problem}


\begin{Problem}
Let $A_n$ be the set of all positive integers divisible by $n$. Find the sets

\begin{enumerate}[a)]
  \item $\displaystyle{\bigcup_{n=2}^\infty A_n}$
  \item $\displaystyle \bigcap_{n=2}^\infty A_n$.
\end{enumerate}

\problemAnswer{
\begin{enumerate}[a)]
  \item Our set here is the set of positive integers which are divisible by any integer $n \ge 2$, which is $\mathbb{N}\setminus\{1\}$.
  \item Here the set contains positive integers which must be divisible by every integer $n \ge 2$. Such integers would be divisible by an infinite number of primes, which is absurd, so the answer is $\emptyset$.
\end{enumerate}
}
\end{Problem}

\begin{Problem}
Find the sets

\begin{enumerate}[a)]
  \item $\displaystyle{\bigcup_{n=1}^\infty \left[a+\frac{1}{n},b-\frac{1}{n} \right]}$
  \item $\displaystyle \bigcap_{n=1}^\infty \left(a-\frac{1}{n},b+\frac{1}{n}\right)$
\end{enumerate}

\problemAnswer{
\begin{enumerate}[a)]
  \item Since $\left[ a+\frac{1}{n}, b-\frac{1}{n} \right] \subset \left[ a+\frac{1}{m}, b-\frac{1}{m} \right]$ if $n \le m$, it suffices to consider
  $$\bigcup_{n=1}^\infty \left[ a+\frac{1}{n}, b-\frac{1}{n} \right] = \lim_{n\rightarrow\infty} \left[ a+\frac{1}{n}, b-\frac{1}{n} \right] = \left( a, b \right)$$.
  \item Similarly, here $\left(a-\frac{1}{n},b+\frac{1}{n}\right) \supset \left(a-\frac{1}{m},b+\frac{1}{m}\right)$ if $n \leq m$, so
$$ \bigcap_{n=1}^\infty \left(a-\frac{1}{n},b+\frac{1}{n}\right) = \lim_{n\rightarrow\infty} \left(a-{1 \over n}, b + {1 \over n}\right) = \left[a,b\right]$$
\end{enumerate}
}
\end{Problem}


\begin{Problem}
Let $A_\alpha$ be the set of points lying on the curve
$$y = \frac{1}{x^\alpha} \ \ \ \ (0 < x < \infty).$$
What is
$$\bigcap_{\alpha \ge 1} A_\alpha \ \rm{?}$$

\problemAnswer{
We consider any two curves $A_\alpha$ and $A_\beta$, where, $\alpha \neq \beta$, and $\alpha,\beta \ge 1$. A point $(a,a^{-\alpha}) \in A_\alpha$ is a member of $A_\beta$ only if $a^{-\alpha} = a^{-\beta}$. But then we must have $a = 1$, and so our result is
$$\bigcap_{\alpha \ge 1} A_\alpha = (1,1).$$
}
\end{Problem}

\begin{Problem}
Let $y = f(x) = \langle x \rangle$ for all real $x$, where $\langle x \rangle$ is the fractional part of $x$. Prove that every closed interval of length 1 has the same image under $f$. What is this image? If $f$ one-to-one? What is the pre-image of the interval $\frac{1}{4} \le y \le \frac{3}{4}$? Partition the real line into classes of points with the same image.\\

\problemAnswer{
We choose an interval $I = [a, a + 1]$ and partition $I$ into two intervals, $I = L \cup R$, where
$$L = [a, a + 1 - \langle a\rangle)$$
$$R = [a + 1 - \langle a\rangle, a+1].$$
We observe that $f(L) = [\langle a \rangle, 1)$ and $f(R) = [0,\langle a \rangle]$. In other words the image of the unit interval is
$$ f(I) = f(L \cup R) = f(L) \cup f(R) = [0,1).$$
Since every unit interval maps to $[0,1)$, $f:\mathbb{R} \to [0,1)$ is not one-to-one. We have that
$$f^{-1} \left[\frac{1}{4}, \frac{3}{4}\right] = \bigcup_{n \in \mathbb{Z}} \left[\frac{1}{4}, \frac{3}{4}\right].$$
In general we can partition the real line into a family of sets $A_\alpha$, where $\alpha \in [0,1)$, such that
$$A_\alpha = \{a \mid a = \alpha +n,\ n \in \mathbb{Z}\}$$
}
\end{Problem}

\begin{Problem}
Given a set $M$, let $\mathcal{R}$ be the set of all ordered pairs of of the form $(a,a)$ where $a \in M$, and let $aRb$ if and only if $(a,b) \in \mathcal{R}$. Interpret the relation $R$.\\

\problemAnswer{
An interpretation of $R$ is simple, since it is just the equality relation restricted to the elements of the set $M$. To see this we note that $(a,b) \in \mathcal{R}$ if and only if $a = b$.
}
\end{Problem}


\begin{Problem}
Give and example of a binary relation which is

\begin{enumerate}[a)]
 \item Reflexive and symmetric, but not transitive;
 \item Reflexive, but neither symmetric nor transitive;
 \item Symmetric, but neither reflexive nor transitive;
 \item Transitive, but neither reflexive nor symmetric.
\end{enumerate}

\problemAnswer{
\begin{enumerate}[a)]
  \item $$aRb \Leftrightarrow \rm{gcd}(a,b) \ne 1$$
    $R$ is reflexive; $\rm{gcd}(a,a) = a \ne 1$. The
    rule $\rm{gdc}(a,b) = \rm{gdc}(b,a)$ implies that
    we have symmetry. Finally, for the transitive property
    we see that for example $20R6$ and $6R21$ together do not imply $20R21$.
  \item $$aRb \Leftrightarrow a \le b \vee a = 0 \vee b = 0$$
    $R$ is reflexive and non symmetric by definition. $R$ is not transitive since we have that $aR0$ and $0Rb$ for any $a > b > 0$, while $aRb$ does not hold.
  \item $$aRb \Leftrightarrow \rm{gcd}(a,b) = 1$$
        This relation is the complement of a) so it is not reflexive, while preserving symmetry. Consider numbers $a$, $b$ such that $\rm{gcd}(a,b)=1$. Since $aRb$ we have $bRa$, but $aRa$ is false, so $R$ is not transitive.
  \item The relation is $>$ on the real numbers. It's only transitive and neither reflexive nor symmetric.
\end{enumerate}
}
\end{Problem}


\end{spacing}
\end{document}

%%%%%%%%%%%%%%%%%%%%%%%%%%%%%%%%%%%%%%%%%%%%%%%%%%%%%%%%%%%%%
