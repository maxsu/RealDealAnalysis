\documentclass{article}
% Change "article" to "report" to get rid of page number on title page
\usepackage{amsmath,amsfonts,amsthm,amssymb}
\usepackage{setspace}
\usepackage{Tabbing}
\usepackage{fancyhdr}
\usepackage{lastpage}
\usepackage{extramarks}
\usepackage{chngpage}
\usepackage{soul,color}
\usepackage{graphicx,float,wrapfig}
\usepackage[shortlabels]{enumitem}

% In case you need to adjust margins:
\topmargin=-0.45in      %
\evensidemargin=0in     %
\oddsidemargin=0in      %
\textwidth=6.5in        %
\textheight=9.0in       %
\headsep=0.25in         %

% Homework Specific Information
\newcommand{\Title}{Introductory Real Analyis Exercises}
\newcommand{\Assignment}{Chaper 2.1}
\newcommand{\BookAuthors}{A.Kolmogorov \& S.Fomin}
\newcommand{\AuthorName}{Max Suica}

% Setup the header and footer
\pagestyle{fancy}                                                       %
\lhead{\AuthorName}                                                     %
\chead{\Title\ : \Assignment}                                           %
\rhead{\firstxmark}                                                     %
\lfoot{\lastxmark}                                                      %
\cfoot{}                                                                %
\rfoot{Page\ \thepage\ of\ \pageref{LastPage}}                          %
\renewcommand\headrulewidth{0.4pt}                                      %
\renewcommand\footrulewidth{0.4pt}                                      %

% This is used to trace down (pin point) problems
% in latexing a document:
%\tracingall

%%%%%%%%%%%%%%%%%%%%%%%%%%%%%%%%%%%%%%%%%%%%%%%%%%%%%%%%%%%%%
% Some tools
\newcommand{\enterProblemHeader}[1]{\nobreak\extramarks{#1}{#1 continued on next page\ldots}\nobreak%
                                    \nobreak\extramarks{#1 (continued)}{#1 continued on next page\ldots}\nobreak}%
\newcommand{\exitProblemHeader}[1]{\nobreak\extramarks{#1 (continued)}{#1 continued on next page\ldots}\nobreak%
                                   \nobreak\extramarks{#1}{}\nobreak}%

\newlength{\labelLength}
\newcommand{\labelAnswer}[2]
  {\settowidth{\labelLength}{#1}%
   \addtolength{\labelLength}{0.25in}%
   \changetext{}{-\labelLength}{}{}{}%
   \noindent\begin{minipage}[c]{\columnwidth}#2\end{minipage}%
   \marginpar{#1}%

   % We put the blank space above in order to make sure this
   % \marginpar gets correctly placed.
   \changetext{}{+\labelLength}{}{}{}}%

\setcounter{secnumdepth}{0}
\newcommand{\ProblemName}{}%
\newcounter{ProblemCounter}%
\newenvironment{Problem}[1][Problem \arabic{ProblemCounter}]%
  {\stepcounter{ProblemCounter}%
   \renewcommand{\ProblemName}{#1}%
   \section{\ProblemName}%
   \enterProblemHeader{\ProblemName}}%
  {\exitProblemHeader{\ProblemName}}%

\newcommand{\problemAnswer}[1]
  {\ \\ \noindent\ \begin{minipage}[c]{\columnwidth}#1\end{minipage}}%

\newcommand{\problemLAnswer}[1]
  {\labelAnswer{\ProblemName}{#1}}

\newcommand{\SectionName}{}%
\newlength{\SectionLabelLength}{}%
\newenvironment{Section}[1]%
  {% We put this space here to make sure we're not connected to the above.
   % Otherwise the changetext can do funny things to the other margin

   \renewcommand{\SectionName}{#1}%
   \settowidth{\SectionLabelLength}{\SectionName}%
   \addtolength{\SectionLabelLength}{0.25in}%
   \changetext{}{-\SectionLabelLength}{}{}{}%
   \subsection{\SectionName}%
   \enterProblemHeader{\ProblemName\ [\SectionName]}}%
  {\enterProblemHeader{\ProblemName}%

   % We put the blank space above in order to make sure this margin
   % change doesn't happen too soon (otherwise \sectionAnswer's can
   % get ugly about their \marginpar placement.
   \changetext{}{+\SectionLabelLength}{}{}{}}%

\newcommand{\sectionAnswer}[1]
  {% We put this space here to make sure we're disconnected from the previous
   % passage

   \noindent\begin{minipage}[c]{\columnwidth}#1\end{minipage}%
   \enterProblemHeader{\ProblemName}\exitProblemHeader{\ProblemName}%
   \marginpar{\SectionName}%

   % We put the blank space above in order to make sure this
   % \marginpar gets correctly placed.
   }%

%%%%%%%%%%%%%%%%%%%%%%%%%%%%%%%%%%%%%%%%%%%%%%%%%%%%%%%%%%%%%


%%%%%%%%%%%%%%%%%%%%%%%%%%%%%%%%%%%%%%%%%%%%%%%%%%%%%%%%%%%%%
% Make title
\title{\vspace{2in}\textmd{\textbf{\Title}}
\\\normalsize\vspace{0.1in}{\BookAuthors}
\\\vspace{0.1in}\large{\textit{\ }}\vspace{3in}}
\date{}
\author{\textbf{\AuthorName}}
%%%%%%%%%%%%%%%%%%%%%%%%%%%%%%%%%%%%%%%%%%%%%%%%%%%%%%%%%%%%%

\begin{document}
\begin{spacing}{1.1}

\maketitle
\newpage
% Uncomment the \tableofcontents and \newpage lines to get a Contents page
% Uncomment the \setcounter line as well if you do NOT want subsections
%       listed in Contents
%\setcounter{tocdepth}{1}
%\tableofcontents
%\newpage

% When problems are long, it may be desirable to put a \newpage or a
% \clearpage before each Problem environment

\clearpage

%%%%%%%%%%%%%%%%%%%%%%%%%%%%%%%%%%%%%%%%%%%%%%%%%%%%%%%%%%%%

\section{\Assignment\ Exercises}


\begin{Problem} Given a metric space $(X,\rho)$, prove that 

\begin{enumerate}[a)]
	\item $|\rho(x,z) - \rho(y,u)|\le \rho(x,y) +\rho(z,u) \quad (x,y,z,y\in X);$
	\item $|\rho(x,z)-\rho(y,z)| \le \rho(x,y) \quad (x,y, z \in X).$
\end{enumerate}

\problemAnswer{

}
\end{Problem}

\begin{Problem}
Verify that
\[\left(\sum_{k=1}^n a_k b_k\right)^2 = \sum_{k=1}^n a_k^2 \sum_{k=1}^n b_k^2- \frac{1}{2}\sum_{i=1}^n\sum_{j=1}^n(a_i b_j - a_j b_i)^2.\]
Deduce the Cauchy-Schwarz inequality from this identity.

\problemAnswer{

}
\end{Problem}

\begin{Problem}
Verify that
\[\left(\int_a^b x(t)y(t) dt \right)^2 = \int_a^b x^2(t)dt \int_a^b y^2(t)dt - \frac{1}{2} \int_a^b \int_a^b[x(s)y(t) - x(t)y(s)]^2 ds dt.\]

Deduce Schwarz's inequality (11) from this identity.
\problemAnswer{

}
\end{Problem}

\begin{Problem}
What goes wrong in Example 10, p. 41 if $p < 1$?
\textit{Hint.} Show that Minkowski's identity fails for $p<1$.
\problemAnswer{

}
\end{Problem}

\begin{Problem}
Prove that the metric (5) is the limiting case of the metric (13) in the sense that
\[\rho_o(x,y) =\max_{1\le k \le n} |x_k - y_k| = \lim_{p \to \infty} \left(\sum_{k=1}^n |x_k - y_k|^p\right)^{1/p}.\]
\problemAnswer{

}
\end{Problem}

\begin{Problem}
Starting from the inequality (19), deduce Holder's integral inequality
\[\int_a^b x(t)y(t) dt\le \left( \int_a^b |x(t)|^p dt\right)^{1/p}\left(\int_a^b|y(t)|^q dt\right)^{1/q}\quad\left(\frac{1}{p} + \frac{1}{q}=1\right),\]
valid for any functions $x(t)$ and $y(t)$ such that the integrals on the right exist.
\problemAnswer{

}
\end{Problem}

\begin{Problem}
Use Holders integral inequality to prove Minkowski's integral inequality
\[\left(\int_a^b|x(t) + y(t)|^p dt \right)^{1/p} \le \left(\int_a^b|x(t)|^p\right)^{1/p} + \left(\int_a^b|y(t)|^p dt\right)^{1/p} \quad (p \le 1).\]
\problemAnswer{

}
\end{Problem}

\end{spacing}
\end{document}

%%%%%%%%%%%%%%%%%%%%%%%%%%%%%%%%%%%%%%%%%%%%%%%%%%%%%%%%%%%%%
