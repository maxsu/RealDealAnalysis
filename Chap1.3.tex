\documentclass{article}
% Change "article" to "report" to get rid of page number on title page
\usepackage{amsmath,amsfonts,amsthm,amssymb}
\usepackage{setspace}
\usepackage{Tabbing}
\usepackage{fancyhdr}
\usepackage{lastpage}
\usepackage{extramarks}
\usepackage{chngpage}
\usepackage{soul,color}
\usepackage{graphicx,float,wrapfig}
\usepackage[shortlabels]{enumitem}

% In case you need to adjust margins:
\topmargin=-0.45in      %
\evensidemargin=0in     %
\oddsidemargin=0in      %
\textwidth=6.5in        %
\textheight=9.0in       %
\headsep=0.25in         %

% Homework Specific Information
\newcommand{\Title}{Introductory Real Analyis Exercises}       %
\newcommand{\Assignment}{Chaper 1.3}                           %
\newcommand{\BookAuthors}{A.Kolmogorov \& S.Fomin}             %
\newcommand{\AuthorName}{Max Suica}                            %

% Setup the header and footer
\pagestyle{fancy}                                                       
\lhead{\AuthorName}                                                    
\chead{\Title\ : \Assignment}                                           
\rhead{\firstxmark}                                                     
\lfoot{\lastxmark}                                                     
\cfoot{}                                                                
\rfoot{Page\ \thepage\ of\ \pageref{LastPage}}                          
\renewcommand\headrulewidth{0.4pt}                                     
\renewcommand\footrulewidth{0.4pt}                                      

% This is used to trace down (pin point) problems
% in latexing a document:
%\tracingall

%%%%%%%%%%%%%%%%%%%%%%%%%%%%%%%%%%%%%%%%%%%%%%%%%%%%%%%%%%%%%
% Some tools
\newcommand{\enterProblemHeader}[1]{\nobreak\extramarks{#1}{#1 continued on next page\ldots}\nobreak%
                                    \nobreak\extramarks{#1 (continued)}{#1 continued on next page\ldots}\nobreak}%
\newcommand{\exitProblemHeader}[1]{\nobreak\extramarks{#1 (continued)}{#1 continued on next page\ldots}\nobreak%
                                   \nobreak\extramarks{#1}{}\nobreak}%

\newlength{\labelLength}
\newcommand{\labelAnswer}[2]
  {\settowidth{\labelLength}{#1}%
   \addtolength{\labelLength}{0.25in}%
   \changetext{}{-\labelLength}{}{}{}%
   \noindent\begin{minipage}[c]{\columnwidth}#2\end{minipage}%
   \marginpar{#1}%

   % We put the blank space above in order to make sure this
   % \marginpar gets correctly placed.
   \changetext{}{+\labelLength}{}{}{}}%

\setcounter{secnumdepth}{0}
\newcommand{\ProblemName}{}%
\newcounter{ProblemCounter}%
\newenvironment{Problem}[1][Problem \arabic{ProblemCounter}]%
  {\stepcounter{ProblemCounter}%
   \renewcommand{\ProblemName}{#1}%
   \section{\ProblemName}%
   \enterProblemHeader{\ProblemName}}%
  {\exitProblemHeader{\ProblemName}}%

\newcommand{\problemAnswer}[1]
  {\ \\ \noindent\ \begin{minipage}[c]{\columnwidth}#1\end{minipage}}%

\newcommand{\problemLAnswer}[1]
  {\labelAnswer{\ProblemName}{#1}}

\newcommand{\SectionName}{}%
\newlength{\SectionLabelLength}{}%
\newenvironment{Section}[1]%
  {% We put this space here to make sure we're not connected to the above.
   % Otherwise the changetext can do funny things to the other margin

   \renewcommand{\SectionName}{#1}%
   \settowidth{\SectionLabelLength}{\SectionName}%
   \addtolength{\SectionLabelLength}{0.25in}%
   \changetext{}{-\SectionLabelLength}{}{}{}%
   \subsection{\SectionName}%
   \enterProblemHeader{\ProblemName\ [\SectionName]}}%
  {\enterProblemHeader{\ProblemName}%

   % We put the blank space above in order to make sure this margin
   % change doesn't happen too soon (otherwise \sectionAnswer's can
   % get ugly about their \marginpar placement.
   \changetext{}{+\SectionLabelLength}{}{}{}}%

\newcommand{\sectionAnswer}[1]
  {% We put this space here to make sure we're disconnected from the previous
   % passage

   \noindent\begin{minipage}[c]{\columnwidth}#1\end{minipage}%
   \enterProblemHeader{\ProblemName}\exitProblemHeader{\ProblemName}%
   \marginpar{\SectionName}%

   % We put the blank space above in order to make sure this
   % \marginpar gets correctly placed.
   }%

%%%%%%%%%%%%%%%%%%%%%%%%%%%%%%%%%%%%%%%%%%%%%%%%%%%%%%%%%%%%%


%%%%%%%%%%%%%%%%%%%%%%%%%%%%%%%%%%%%%%%%%%%%%%%%%%%%%%%%%%%%%
% Make title
\title{\vspace{2in}\textmd{\textbf{\Title}}
\\\normalsize\vspace{0.1in}{\BookAuthors}
\\\vspace{0.1in}\large{\textit{\ }}\vspace{3in}}
\date{}
\author{\textbf{\AuthorName}}
%%%%%%%%%%%%%%%%%%%%%%%%%%%%%%%%%%%%%%%%%%%%%%%%%%%%%%%%%%%%%

\begin{document}
\begin{spacing}{1.1}

\maketitle
\newpage
% Uncomment the \tableofcontents and \newpage lines to get a Contents page
% Uncomment the \setcounter line as well if you do NOT want subsections
%       listed in Contents
%\setcounter{tocdepth}{1}
%\tableofcontents
%\newpage

% When problems are long, it may be desirable to put a \newpage or a
% \clearpage before each Problem environment

\clearpage

%%%%%%%%%%%%%%%%%%%%%%%%%%%%%%%%%%%%%%%%%%%%%%%%%%%%%%%%%%%%

\section{\Assignment\ Exercises}


\begin{Problem}
Exhibit both a partial ordering and a simple ordering of the set of all complex numbers.

\problemAnswer{
There is the trivial partial ordering 
$$a \le b \Leftrightarrow a = b,$$
a kind of partial ordering by rays from the origin where nonzero points on different rays are incomparable, and zero is the minimal element 
$$a \le b \Leftrightarrow a = sb, s \in [0,1],$$
another ordering that compares numbers equidistant to the origin according to their argument
$$a \le b \Leftrightarrow |a| = |b| \wedge \rm{arg}(a) \le \rm{arg}(b),$$
and one last example
$$ a \le b \Leftrightarrow \Re (a) = \Re(b) \wedge \Im(a) \le \Im(b).$$

Finally, as an example of a simple ordering, consider
$$a \le b \Leftrightarrow \Im(a) < \Im(b) \vee \left( \Im(a) = \Im(b) \wedge \Re(a) \le \Re(b) \right). $$
}
\end{Problem}

\begin{Problem}
What is the minimal element of the set of all subset of a given set $X$, partially ordered by set inclusion? What is the maximal element?

\problemAnswer{
In $(2^M, \subset)$ a minimal element is $\emptyset$ and a maximal element is $M$ itself. To see that the maximal element is unique suppose that there is another maximimal element $M'$ and construct $S = M \cup M'$. Then $M \subset S$ and $M' \subset S$ together imply $M = S = M'$. A similar argument shows $\emptyset$'s uniqueness.
}
\end{Problem}

\begin{Problem}
A partially ordered set $M$ is said to be a \textit{directed set} if, given any two elements $a,b \in M$, there is an element $c \in M$ such that $a \le c$, $b\le c$. Are the partially ordered sets in Examples 1-4, Sec 3.1 all directed sets?

\problemAnswer{
\begin{enumerate}
	\item The trivial partial ordering on a set is directed. Any element $a$ is comparable only to itself, so this is not directed.
	\item The continuous functions on $[\alpha, \beta]$, ordered by $f \le g \Leftrightarrow f(t) \ge g(t)$ for all $t \in [\alpha, \beta]$. Given two continuous functions $f$ and  $g$ we have $f(t) \le \rm{max}(f(t), g(t))$ and similar for $g$, so this set is directed.
	\item The subsets of a set $M$, ordered by inclusion. For $A,B \in 2^M$, $A \subset M$ and $B \subset M$, so this is a directed set.
	\item The integers greater than 1, ordered by divisibility. For numbers $a, b > 1$, we have $a \le ab$ and $b \le ab$, so this is a directed set.
\end{enumerate}
}
\end{Problem}

\begin{Problem}
Prove that the set of all subsets of a given set $X$, ordered by set inclusion, is a lattice. What is the set theoretic meaning of the greatest lower bound and least upper bound of two elements of this set?

\problemAnswer{
Let $A$ and $B$ be subsets of $X$.

\begin{enumerate}[a)]
  \item I claim that the least upper bound of $A$ and $B$ under inclusion is $L = A \cup B$. Suppose there is another least upper bound $L'$. Then $A \subset L'$ and $B \subset L'$, which means that $A \cup B \subset L'$ and thus $L'$ must be equal to $L$ or else it would be greater.

  \item I also claim that $G = A \cap B$ is the greatest lower bound of $A$ and $B$. To show this suppose there is some other greatest lower bound $G'$. Then $G' \subset A$ and $G' \subset B$ but then $G' \subset G$ so $G' = G$.
\end{enumerate}

Thus the greatest lower bound and lowest upper bound for two subsets of a set are exactly the operations of set intersection and union, respectively.
}
\end{Problem}

\begin{Problem}
Prove that an order preserving mapping of one ordered set onto another is automatically an isomorphism.

\problemAnswer{
Since both sets are ordered then each element $a$ is comparable to any other element $b$. But then for any $f(a)$ and $f(b)$ where $f(a) \le f(b)$ we must have that $a \le b$, since if $b \le a$ then $f$ would not be order preserving.
}
\end{Problem}

\begin{Problem}
Prove that ordered sums and products of ordered sets are associative.

\problemAnswer{
Let $A$, $B$ and $C$ be disjoint ordered sets. Then the order relation on $(A+B)+C$ is
\[ a \le b \Leftrightarrow \ (a \le_A b) \vee (a \le_B b) \vee (a\in A \wedge b \in B) \vee\ (a \le_C b) \vee (a \in A \cup B \wedge b \in C).\]

On $A + (B+C)$ the relation is
\[ a \le b \Leftrightarrow (a \le_A b) \vee (a \le_B b) \vee (a \le_C b) \vee (a \in B \wedge b \in C) \vee (a \in A \wedge a \in B \cup C) .\]

But it's easy to see that both of these can be turned into
\[ a \le b \Leftrightarrow \ (a \le_A b) \vee (a \le_B b) \vee\ (a \le_C b) \vee (a\in A \wedge b \in B) \vee (a \in A \wedge b \in C) \vee (a \in B \wedge b \in C).\]

To show that the ordered product is associative, let $A$, $B$, $C$ again be disjoint ordered sets. The order relation on $(A \times B) \times C$ is

\[(a_1, a_2, a_3) \le (b_1, b_2, b_3) \Leftrightarrow (a_1 \le b_1) \vee (a_1 = a_2 \wedge b_1 \le b_2) \vee ((a_1, a_2) =  (b_1,b_2) \wedge a_3 \le b_3). \]

For $A \times (B \times C)$ the relation is

\[(a_1, a_2, a_3) \le (b_1, b_2, b_3) \Leftrightarrow (a_1 \le b_1) \vee (a_1 = b_1 \wedge (a_2 \le b_2 \vee (a_2 = b_2 \wedge a_3 \le b_3))). \]

But these both expand to

\[(a_1, a_2, a_3) \le (b_1, b_2, b_3) \Leftrightarrow (a_1 \le b_1) \vee (a_1 = b_1 \wedge a_2 \le b_2) \vee (a_1 = b_1 \wedge a_2 = b_2 \wedge a_3 \le b_3). \]

         
}
\end{Problem}

\begin{Problem}
Construct well ordered sets with ordinals
$$\omega + n, \omega + \omega, \omega + \omega + n, \omega + \omega + \omega, \ldots$$

\problemAnswer{
The set $\{n+1, n+2, \ldots \} + \{1,2,\ldots,n\}$ is of the first order type.\\
 The set $\{1, 3, 5,\ldots\} + \{2, 4, 6, \ldots\}$ is of the second order type.

In general $\omega + \omega + \ldots + \omega = m \times \omega$. We can construct a set with ordinal $m \times \omega + n$ as

\[ \{1,1+m,1+2m,\ldots \}+\{2,2+m,\ldots\}+\ldots+\{m,2m,3m,\ldots\}+\{-1,-2,\ldots, -n\}  \]
}
\end{Problem}

\begin{Problem}
Construct well ordered sets with ordinals
$$\omega \cdot n, \omega^2, \omega^2 \cdot n, \omega^3, \ldots$$
Show that the sets are all countable.

\problemAnswer{

}
\end{Problem}

\begin{Problem}
Show that 
$$\omega + \omega = \omega \cdot 2, \omega + \omega + \omega = \omega \cdot 3, \ldots$$

\problemAnswer{

}
\end{Problem}

\begin{Problem}
Prove that the set $W(\alpha)$ of all ordinals less that a given ordinal $\alpha$ is well-ordered.

\problemAnswer{

}
\end{Problem}

\begin{Problem}
Prove that any nonempty set of ordinals is well-ordered.

\problemAnswer{

}
\end{Problem}

\begin{Problem}
Prove that the set $M$ of all ordinals corresponding to a countable set is itself uncountable.

\problemAnswer{
Suppose that $M$ is countable. Then the $M$'s ordinal $\alpha$ must be a memeber of $M$.
}
\end{Problem}

\begin{Problem}
Let $\aleph_1$ be the power of the set $M$ in the preceding problem. Prove that there is no power $m$ such that $\aleph_0 < m < \aleph_1$.

\problemAnswer{

}
\end{Problem}

\end{spacing}
\end{document}

%%%%%%%%%%%%%%%%%%%%%%%%%%%%%%%%%%%%%%%%%%%%%%%%%%%%%%%%%%%%%
