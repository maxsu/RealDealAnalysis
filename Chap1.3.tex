\documentclass{article}
% Change "article" to "report" to get rid of page number on title page
\usepackage{amsmath,amsfonts,amsthm,amssymb}
\usepackage{setspace}
\usepackage{Tabbing}
\usepackage{fancyhdr}
\usepackage{lastpage}
\usepackage{extramarks}
\usepackage{chngpage}
\usepackage{soul,color}
\usepackage{graphicx,float,wrapfig}
\usepackage[shortlabels]{enumitem}

% In case you need to adjust margins:
\topmargin=-0.45in      %
\evensidemargin=0in     %
\oddsidemargin=0in      %
\textwidth=6.5in        %
\textheight=9.0in       %
\headsep=0.25in         %

% Homework Specific Information
\newcommand{\Title}{Introductory Real Analyis Exercises}       %
\newcommand{\Assignment}{Chaper 1.3}                           %
\newcommand{\BookAuthors}{A.Kolmogorov \& S.Fomin}             %
\newcommand{\AuthorName}{Max Suica}                            %

% Setup the header and footer
\pagestyle{fancy}                                                       
\lhead{\AuthorName}                                                    
\chead{\Title\ : \Assignment}                                           
\rhead{\firstxmark}                                                     
\lfoot{\lastxmark}                                                     
\cfoot{}                                                                
\rfoot{Page\ \thepage\ of\ \pageref{LastPage}}                          
\renewcommand\headrulewidth{0.4pt}                                     
\renewcommand\footrulewidth{0.4pt}                                      

% This is used to trace down (pin point) problems
% in latexing a document:
%\tracingall

%%%%%%%%%%%%%%%%%%%%%%%%%%%%%%%%%%%%%%%%%%%%%%%%%%%%%%%%%%%%%
% Some tools
\newcommand{\enterProblemHeader}[1]{\nobreak\extramarks{#1}{#1 continued on next page\ldots}\nobreak%
                                    \nobreak\extramarks{#1 (continued)}{#1 continued on next page\ldots}\nobreak}%
\newcommand{\exitProblemHeader}[1]{\nobreak\extramarks{#1 (continued)}{#1 continued on next page\ldots}\nobreak%
                                   \nobreak\extramarks{#1}{}\nobreak}%

\newlength{\labelLength}
\newcommand{\labelAnswer}[2]
  {\settowidth{\labelLength}{#1}%
   \addtolength{\labelLength}{0.25in}%
   \changetext{}{-\labelLength}{}{}{}%
   \noindent\begin{minipage}[c]{\columnwidth}#2\end{minipage}%
   \marginpar{#1}%

   % We put the blank space above in order to make sure this
   % \marginpar gets correctly placed.
   \changetext{}{+\labelLength}{}{}{}}%

\setcounter{secnumdepth}{0}
\newcommand{\ProblemName}{}%
\newcounter{ProblemCounter}%
\newenvironment{Problem}[1][Problem \arabic{ProblemCounter}]%
  {\stepcounter{ProblemCounter}%
   \renewcommand{\ProblemName}{#1}%
   \section{\ProblemName}%
   \enterProblemHeader{\ProblemName}}%
  {\exitProblemHeader{\ProblemName}}%

\newcommand{\problemAnswer}[1]
  {\ \\ \noindent\ \begin{minipage}[c]{\columnwidth}#1\end{minipage}}%

\newcommand{\problemLAnswer}[1]
  {\labelAnswer{\ProblemName}{#1}}

\newcommand{\SectionName}{}%
\newlength{\SectionLabelLength}{}%
\newenvironment{Section}[1]%
  {% We put this space here to make sure we're not connected to the above.
   % Otherwise the changetext can do funny things to the other margin

   \renewcommand{\SectionName}{#1}%
   \settowidth{\SectionLabelLength}{\SectionName}%
   \addtolength{\SectionLabelLength}{0.25in}%
   \changetext{}{-\SectionLabelLength}{}{}{}%
   \subsection{\SectionName}%
   \enterProblemHeader{\ProblemName\ [\SectionName]}}%
  {\enterProblemHeader{\ProblemName}%

   % We put the blank space above in order to make sure this margin
   % change doesn't happen too soon (otherwise \sectionAnswer's can
   % get ugly about their \marginpar placement.
   \changetext{}{+\SectionLabelLength}{}{}{}}%

\newcommand{\sectionAnswer}[1]
  {% We put this space here to make sure we're disconnected from the previous
   % passage

   \noindent\begin{minipage}[c]{\columnwidth}#1\end{minipage}%
   \enterProblemHeader{\ProblemName}\exitProblemHeader{\ProblemName}%
   \marginpar{\SectionName}%

   % We put the blank space above in order to make sure this
   % \marginpar gets correctly placed.
   }%

%%%%%%%%%%%%%%%%%%%%%%%%%%%%%%%%%%%%%%%%%%%%%%%%%%%%%%%%%%%%%


%%%%%%%%%%%%%%%%%%%%%%%%%%%%%%%%%%%%%%%%%%%%%%%%%%%%%%%%%%%%%
% Make title
\title{\vspace{2in}\textmd{\textbf{\Title}}
\\\normalsize\vspace{0.1in}{\BookAuthors}
\\\vspace{0.1in}\large{\textit{\ }}\vspace{3in}}
\date{}
\author{\textbf{\AuthorName}}
%%%%%%%%%%%%%%%%%%%%%%%%%%%%%%%%%%%%%%%%%%%%%%%%%%%%%%%%%%%%%

\begin{document}
\begin{spacing}{1.1}

\maketitle
\newpage
% Uncomment the \tableofcontents and \newpage lines to get a Contents page
% Uncomment the \setcounter line as well if you do NOT want subsections
%       listed in Contents
%\setcounter{tocdepth}{1}
%\tableofcontents
%\newpage

% When problems are long, it may be desirable to put a \newpage or a
% \clearpage before each Problem environment

\clearpage

%%%%%%%%%%%%%%%%%%%%%%%%%%%%%%%%%%%%%%%%%%%%%%%%%%%%%%%%%%%%

\section{\Assignment\ Exercises}


\begin{Problem}
Exhibit both a partial ordering and a simple ordering of the set of all complex numbers.
\problemAnswer{

}
\end{Problem}

\begin{Problem}
What is the minimal element of the set of all subset of a given set $X$, partially ordered by set inclusion? What is the maximal element?
\problemAnswer{

}
\end{Problem}

\begin{Problem}
A partially ordered set $M$ is said to be a \textit{directed set} if, given any two elements $a,b \in M$, there is an element $c \in M$ such that $a \le c$, $b\le c$. Are the partially ordered sets in Examples 1-4, Sec 3.1 all directed sets?
\problemAnswer{

}
\end{Problem}

\begin{Problem}
Prove that the set of all subsets of a given set $X$, ordered by set inclusion, is a lattice. What is the set theoretic meaning of the greatest lower bound and least upper bound of two elements of this set?
\problemAnswer{

}
\end{Problem}

\begin{Problem}
Prove that an order preserving mapping of one ordered set onto another is automatically an isomorphism.
\problemAnswer{

}
\end{Problem}

\begin{Problem}
Prove that ordered sums and products of ordered sets are associative.
\problemAnswer{

}
\end{Problem}

\begin{Problem}
Construct well ordered sets with ordinals
$$\omega + n, \omega + \omega, \omega + \omega + n, \omega + \omega + \omega, \ldots$$
\problemAnswer{

}
\end{Problem}

\begin{Problem}
Construct well ordered sets with ordinals
$$\omega \cdot n, \omega^2 \cdot n, \omega^3, \ldots$$
Show that the sets are all countable.
\problemAnswer{

}
\end{Problem}

\begin{Problem}
Show that 
$$\omega + \omega = \omega \cdot 2, \omega + \omega + \omega = \omega \cdot 3, \ldots$$
\problemAnswer{

}
\end{Problem}

\begin{Problem}
Prove that the set $W(\alpha)$ of all ordinals less that a given ordinal $\alpha$ is well-ordered.
\problemAnswer{

}
\end{Problem}

\begin{Problem}
Prove that any nonempty set of ordinals is well-ordered.
\problemAnswer{

}
\end{Problem}

\begin{Problem}
Prove that the set $M$ of all ordinals corresponding to a countable set is itself uncountable.
\problemAnswer{

}
\end{Problem}

\begin{Problem}
Let $\aleph_1$ be the power of the set $M$ in the preceding problem. Prove that there is no power $m$ such that $\aleph_0 < m < \aleph_1$.
\problemAnswer{

}
\end{Problem}

\end{spacing}
\end{document}

%%%%%%%%%%%%%%%%%%%%%%%%%%%%%%%%%%%%%%%%%%%%%%%%%%%%%%%%%%%%%
